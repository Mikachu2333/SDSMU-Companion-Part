\part[简要说明]{简要说明}
\begin{enumerate}
    \item 在校生可免费开通以“@stu.sdsmu.edu.cn”结尾的教育邮箱(也称学生邮箱、校园邮箱),毕业时回收
    \item 主要用于:学术交流、投递简历、向科技公司申请学生优惠等
    \item \colored{255}{0}{0}{禁止交流机密信息、外借、泄露信息或参与违反法律法规或相关规章制度的活动}
\end{enumerate}

\part[教程]{教程}

\section[申请邮箱]{申请邮箱}
\begin{enumerate}
    \item 打开“山东第二医科大学”app→点击底部菜单“应用”
    \item 点击“业务申请”→“学生邮箱申请流程”→“我要办理”
    \item 按照要求填写所有信息并仔细阅读说明→提交信息,等待通过
    \item 提交后的下个月1号自动注册,进入下一小节的首次登录教程
\end{enumerate}

\section[激活邮箱(首次登录)]{激活邮箱(首次登录)}
\textbf{必须使用电脑端浏览器进行!}

\begin{enumerate}
    \item 在提交信息后的下个月第一天后(推荐2~9号)方可进行此小节步骤,账号创建10天后逾期未登录视为放弃账号申请
    \item 打开\uhref{https://edu.icoremail.net}{https://edu.icoremail.net} →点击“忘记密码”→
    \item 选择“山东第二医科大学”→按照步骤重置密码(相当于首次设置密码)→\\
          注:密码建议14~16位,含大小写、数字、特殊字符如下划线“\_”等
    \item 再次打开\uhref{https://edu.icoremail.net}{https://edu.icoremail.net}(或\uhref{https://mail.stu.sdsmu.edu.cn}{https://mail.stu.sdsmu.edu.cn}),并登录
\end{enumerate}

\section[设置邮箱]{设置邮箱}

\subsection[常规设置]{常规设置}

\subsubsection[关闭“反垃圾邮件”功能]{关闭“反垃圾邮件”功能\footnotemark}
\footnotetext{因此功能经常导致无法收到非联系人发送的邮件,因此推荐关闭。}
\textbf{必须使用电脑端浏览器进行}
\begin{enumerate}
    \item 在邮件主页面单击最左下角的“设置”→安全设置→
    \item “反垃圾机制”→调整选项为“关闭”
\end{enumerate}

\subsubsection[设置发信人名称]{设置发信人名称\footnotemark}
\footnotetext{根据需要取消,若不取消,则发件人显示为真实姓名。}
\textbf{必须使用电脑端浏览器进行}
\begin{enumerate}
    \item 打开左下角“设置”→“收发信设置”→
    \item “写信设置”→勾选或取消“发件人显示我的姓名:×××”选项即可
\end{enumerate}

\subsubsection[关联邮箱至“山东第二医科大学”app]{关联邮箱至“山东第二医科大学”app}
\textbf{必须使用电脑端浏览器进行}
\begin{enumerate}
    \item 打开\uhref{https://cas.sdsmu.edu.cn:4102}{安全中心(智慧校园系统)}→点击上方菜单“账户安全”→“安全设置”→
    \item 点击“绑定/解绑邮箱”→按要求填写要绑定的校园邮箱→确认后刷新页面
    \item 等待几天后再使用学校app即可正常查看邮件
\end{enumerate}
\subsection[安全设置]{安全设置}

\subsubsection[开启二次验证]{开启二次验证}
\textbf{必须使用电脑端浏览器进行}
\begin{enumerate}
    \item 点击左下角“设置”→“安全设置”→切换至“二次验证设置”→
    \item 推荐同时开启“短信”、“备用邮箱”、“微信”三项,按提示直接操作即可→
    \item 如果了解OTP并知道如何开启、如何调整可以实验性地开启此选项
\end{enumerate}

\subsubsection[设置备用邮箱]{设置备用邮箱}
\textbf{必须使用电脑端浏览器进行}

与二次验证在同一页面,略

\subsubsection[生成“客户端专用密码”]{生成“客户端专用密码”}
\textbf{必须使用电脑端浏览器进行}
\begin{enumerate}
    \item 本密码用于开启二次验证后,使用其他设备接收、发送邮件;各软件各设备应使用不同的客户端密码
    \item 点击左下角“设置”→“安全设置”→切换至“客户端专用密码”→
    \item 点击“生成专用密码”→输入备注名称→保存生成的密码(\textbf{只显示一次},请务必牢记)
\end{enumerate}

\subsection[配置第三方邮件客户端]{配置第三方邮件客户端}
下文以Thunderbird为例\footnotemark
\footnotetext{以其为例的原因是该软件应用较广泛且配置过程最为繁琐。一般用户推荐使用QQ邮箱、网易邮箱大师等成熟、广泛使用的客户端以减少折腾的麻烦程度。通常国内的邮箱客户端大都可以自动应用设置而无须手动操作,若自动配置失败也可参照本节内容。}

\begin{enumerate}
    \item 首先,参照上方“生成‘客户端专用密码’”小节的内容,将生成的密码保存备用
    \item 切换Thunderbird至“账户管理”页面,点击“添加邮件账户”→
    \item 输入账号、专用密码→点击左下角“高级”→根据下文内容逐一对应填写即可\\
          \textbf{注:网站提示有误,切勿相信}
    \item 配置如下:\\
          \textbf{示例账号:}\\
          用户名:test@stu.sdsmu.edu.cn\\
          专用密码:aaaa aaaa aaaa aaaa(空格为美观需要,实际输入请删除)\\[20pt]
          \noindent\makebox[\linewidth][l]{%
              \begin{minipage}[c][27ex][t]{.5\linewidth}
                  \textbf{收件服务器(收取信件):}\\
                  协议:IMAP\\
                  地址:mail.stu.sdsmu.edu.cn\\
                  端口:993\\
                  安全性:SSL/TLS\\
                  身份验证:普通密码\\
                  用户名:test@stu.sdsmu.edu.cn\\
                  密码:aaaaaaaaaaaaaaaa\\
                  客户端证书:无
              \end{minipage}%
              \begin{minipage}[c][27ex][t]{.5\linewidth}
                  \textbf{发件服务器(发送邮件):}\\
                  协议:SMTP\\
                  地址:mail.stu.sdsmu.edu.cn\\
                  端口:465\\
                  安全性:SSL/TLS\\
                  身份验证:普通密码\\
                  用户名:test@stu.sdsmu.edu.cn\\
                  密码:aaaaaaaaaaaaaaaa\\
                  客户端证书:无
              \end{minipage}%
          }
\end{enumerate}

\section[其他事项]{其他事项}
每180天需要更改一次密码(密码建议14~16位,含大小写、数字、特殊字符如下划线“\_”等)
