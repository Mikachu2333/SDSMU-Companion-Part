\section[总述]{总述}
本指南旨在帮助新生做出一个『清晰明了』的PPT(也称Slide),美观等其他需求敬请自行探索。但无论如何都请把清晰易懂放在第一位,一个花里胡哨且逻辑混乱的PPT只会使你自取其咎。

\section[纲举目张]{纲举目张}
以目的为驱动力制作你的PPT是最快速最有效的(尤其是面对DEADLINE时)
\begin{enumerate}
    \item 我的PPT是给谁看的
          \begin{enumerate}
              \item 老师——体现所有得分要点,娴熟、无误
              \item 同学——尽量有趣(\textbf{有趣不等于插图与视频})
              \item 专家——专业冷峻、摆数据讲道理,翔实具体
              \item 观众——冲击性、能打动人,多预案、演讲者为主
              \item 兼有之——辩证看待主次关系,把握主要矛盾
          \end{enumerate}
    \item 这个PPT的目的是什么
          \begin{enumerate}
              \item 拿高分——要点齐、数据真、不水时间,所有人熟悉PPT中自己负责的部分
              \item 了解一个领域的基本情况——高屋建瓴、忌深究细节、提前准备高概率问题
              \item 说明项目目标——突出前景、强调(或弱化)难点、专业热情、准备充足
          \end{enumerate}
    \item 需要包含哪些内容(下为基本框架,\textbf{需}酌情增减)
          \begin{enumerate}
              \item 概述
              \item 背景
              \item 目的
              \item 内容(各要点清晰明了,\textbf{理清逻辑与递进并列关系})
              \item 结果
              \item 其他
          \end{enumerate}
    \item 以何种方式展示我的想法
          \begin{enumerate}
              \item 文字为主——忌字体难以辨别、忌文字过小、避免使用部分颜色搭配(详见下文)
              \item 图片为主——宜大不宜小、忌动画繁杂、忌图片堆叠
              \item 数据为主——数据较新、来源权威、细致明确、图表清晰字体较大
              \item 视频为主——忌喧宾夺主、忌风格不符、忌生涩难懂
          \end{enumerate}
          \pagebreak[3]
    \item 组内如何分工(可进一步细分或合并)
          \begin{enumerate}
              \item 话题探索\\
                    决定初步方向,同时提供基础的内容框架、论据方向与大体框架
              \item 文稿写作\\
                    细化改进上述框架,遇到需要提供数据的地方暂时跳过并要求相关组员查找所需数据
              \item 论据(资源)查找\\
                    查找相关图片、视频资料或相关观点论据,需频繁查阅论文、相关领域综述等
              \item 数据整理、核实、图标制作\\
                    整理并绘制所有需要的数据与图表,明确所有数据的出处是否可信(自媒体一概排除,应以教材、权威媒体报道、影响因子较高的期刊数据为主)
              \item PPT制作\\
                    只许一人,最多两人(内容>30页时):其一负责所有内容的排布,其二负责所有格式的排版与布局,统一风格
              \item 其他
          \end{enumerate}
    \item 是否有格式要求与其他规定、注意事项
          \begin{enumerate}
              \item 遵守所有规则与要求
              \item 检查排版
              \item 检查其他设置(是否已取消自动放映、特殊字体、图片、视频齐全)
          \end{enumerate}
\end{enumerate}

\section[预先准备]{预先准备}
\begin{enumerate}
    \item 设备要求\\
          电脑(Win10及以上,最低Win7),\textbf{不允许使用手机制作与修改}
    \item 选定软件\\
          使用\textbf{微软Office套件}(2016及以上版本),或\textbf{金山WPS套件}(2018及以上版本),\textbf{禁用}永中Office、LibreOffice、OpenOffice、腾讯文档(及小程序)、金山文档小程序、石墨文档、语雀等,除非你精通电脑使用且可以解决你遭受的所有问题
    \item 内容制作\\
          先制作大体框架,再填充内容,最后排版
    \item 保存
          \begin{enumerate}
              \item 建议『pptx』与『ppt』格式各保存一份以免出现兼容问题
              \item 所有引用的文件(视频、图片)均保存于U盘内(尽量不要使用在线资源以免因教室电脑无法联网而失效)
              \item \textbf{图片应提前压缩}(详见下文),否则文件过大将导致ppt崩溃
              \item 提前检查ppt内所有资源引用无误(可以在其他人电脑上进行放映测试资源是否失效)
          \end{enumerate}
\end{enumerate}

\section[入门指南]{入门指南}
注意:下述字体均以微软雅黑为标准,其他较小或较细字体如楷体、仿宋等需酌情加大字号。

\subsection[明确比例]{明确比例}
\begin{enumerate}
    \item PPT有4:3与16:9两种主流比例,4:3像正方形,16:9是横长方形;下述的各类参数(如:24/32号字体)均以4:3、16:9的顺序进行,可在菜单栏→设计→幻灯片版式(幻灯片大小)处更改
    \item 错误的比例将使ppt字体缩小严重导致前三排也看不清,或放大严重导致图片全部错位遮挡文本,请务必明确投影尺寸
    \item \textbf{当前学校主流投影比例为4:3}
\end{enumerate}

\subsection[首页规范]{首页规范}
\begin{enumerate}
    \item 汇报大标题\\
          字体不应小于80/75号
    \item 其他所有的必要信息,并\textbf{仔细检查是否有误}\\
          含组别、姓名、学号、联系方式,制作日期,课程名称、学期,教师姓名、工号\\
          字体不应小于25/25号
\end{enumerate}

\subsection[目录规范]{目录规范}
\begin{enumerate}
    \item 在第二页含有一个简洁明了的目录
    \item 每个标题不宜超过7字(页面宽度的$\frac{2}{3}$)
    \item 每页小标题数量不宜超过5个(页面长度的$\frac{4}{5}$)
    \item 字体不应小于80/55号,间距不应小于1.5倍
\end{enumerate}

\subsection[内容规范]{内容规范}
\begin{enumerate}
    \item \textbf{PPT内仅应含有所讲内容的关键部分}而非一昧照抄原文
    \item 上下至少各留出两行文本的高度以免投影偏差导致文字消失
    \item 左右至少各留出两个字的宽度
    \item 如内容过多应自行分页,\textbf{严禁为节省页数而缩小字号}
    \item PPT显示的内容与口述补充的内容在7:3或8:2左右最佳
    \item 各类关键数据的引用(例如学术数据、课标外的公式定义等)应当按\\《\textbf{GB/T}\ 7714—2015 信息与文献 参考文献著录规则》的相关标准在结尾\textbf{标明出处}
    \item 普通字体不应小于55/40号,间距不应小于1.2倍,小标题等酌情加大加粗
\end{enumerate}

\subsection[字体与段落规范]{字体与段落规范}
\begin{enumerate}
    \item 中文西文字体各不超过3种,全篇不超5种
    \item \textbf{PPT字体不应过小,间距不应过密},具体标准参照上文
    \item \textbf{禁止}使用文字阴影、艺术字
    \item 中文除宋体、仿宋、楷体、隶书、微软雅黑以外,西文除Arial、Times New Rome以外,所有 \textbf{字体必须内嵌于PPT}(例如艺术字、书法体等,操作步骤可搜索“在PPT内嵌入字体”)
    \item 需要强调的内容应加粗、更换颜色(颜色不应超过4种)
    \item 确定标点符号正确,让中文的归中文,西文的归西文
\end{enumerate}

\subsection[图片规范]{图片规范}
\begin{enumerate}
    \item 每张图片不应大于10㎆\footnotemark
          \footnotetext{如果图片过大,可以使用微信或QQ发送再保存非原图缩小体积,或直接通过\uhref{https://gitee.com/LinkChou/rimage_gui/releases/latest}{Rimage\_GUI}缩小体积(软件开源,如被报毒请自行分辨)。}
    \item 图片主题应明确,切忌花里胡哨、切忌主题不明
    \item 如图片模糊,需手动检查PPT设置,将PPT更改为“禁止自动压缩图片”,操作步骤请自行搜索
    \item 如需在文档中加入思维导图,应按逻辑归属分级绘制并导入图片\\
          例如,对于一张叙述游戏类别的导图,在插入时应这样做:首先,游戏分为沙盒游戏、塔防游戏、益智游戏、单机游戏、网游等,此为一张图片;单击游戏知名的有“保卫萝卜”、“植物大战僵尸”等,这是第二张图片;以此类推,以免图片字体过小无法看清
\end{enumerate}

\subsection[表格规范]{表格规范}
\begin{enumerate}
    \item 应使用三线表
    \item 必要时(如表中插入了大量分式、特殊格式)应将表格导出为图片(截图)再插入
    \item 表格必须含有:序号、标题(前两项在一行内并居中)、表头(必要时含单位,居中)、表身(是否居中依据内容而定)、表注(居左)
\end{enumerate}

\subsection[动画规范]{动画规范}
\begin{enumerate}
    \item \textbf{PPT不应有过多动画以及元素堆叠}\\
          例如,绝对禁止PPT中的一页内含15张大图片,依靠动画一张张切换,以免软件突然崩溃
    \item 动画速度要快,持续时间要短
\end{enumerate}

\subsection[配色规范]{配色规范}
\begin{enumerate}
    \item 请\textbf{尽量使用}以下几种经典的『文字--背景』配色(\st{虽然确实难看}):
          \begin{enumerate}
              \item 黑--白(白--黑)
              \item 红--白
              \item 深蓝--白
              \item 黄--黑(黑--黄)
          \end{enumerate}
    \item \textbf{忌用}以下投影后效果一塌糊涂的配色
          \begin{enumerate}
              \item 橙--白(白--橙)
              \item 亮绿--白(白--亮绿)
              \item 白--红
              \item 红--蓝(蓝--红)
          \end{enumerate}
\end{enumerate}

\subsection[动画规范]{动画规范}
注:下文的后缀名,如『.pptx』为文件\textbf{自带},\textbf{严禁手动改动}。详情百度“如何显示文件后缀”
\begin{enumerate}
    \item 易懂第一,\textbf{含所有必须信息}
    \item 宁长勿短,禁止默认名称
    \item 文件名形式示例:(虽然长了点,但是绝对不会混淆或错交未完工的文件)
          \begin{enumerate}
              \item 在提交时:『××级×班×组关于×××的汇报 2024.12.7(终稿).pptx』
              \item 组内交流修改时:『关于×××(第×版).pptx』
          \end{enumerate}
    \item \textbf{禁用}的文件名示例:
          \begin{enumerate}
              \item 『新建 Microsoft PowerPoint 演示文稿(1)(2)(5).pptx』
              \item 『aaa.ppt』
              \item 『1组.wpt』(wpt是wps的专属格式)
          \end{enumerate}
\end{enumerate}

\subsection[文件保存规范]{文件保存规范}
\begin{enumerate}
    \item \textbf{必须同时}以『.pptx』后缀与『.ppt』后缀各保存一份
    \item \textbf{禁止保存为特殊格式},如“.wpt”、“.odp”等
    \item \textbf{严禁手动修改后缀名}。如需在『.ppt』『.pptx』之间相互转换请使用PowerPoint或WPS等办公软件,点击左上角“文件”菜单--“另存为”按钮;『.wpt』转为上述两种格式需使用WPS,方法同上
\end{enumerate}

\section[进阶指南]{进阶指南}
\subsection[模板的使用]{模板的使用}
模板是个好东西,可以通过套用的方式快速是你的PPT变得“人模狗样”,但是样式与内在毫无关联,它不能改变你的逻辑与论点论据,请遵循『\textbf{内容优先}』的原则完成你的PPT。

你可以使用WPS的模板功能(需要VIP),或是使用iSlide软件(大部分需要VIP),或者自己从网络上下载一个模板进行使用,如果你需要使用模板,请\textbf{先套用模板,再进行编辑}。如果你认为模板的元素不顺眼,只需要模板的背景,你可以删除模板中的所有元素再自定义,但是一定要在编辑完成后点击“放映”,确认无误后与你的组员们相互传输此文件。

在模板的风格方面,请让你的模板风格与演讲(汇报)内容相匹配,例如,不要给《甲状腺癌的分期与预后研究》主题的汇报套一个粉嫩童话的模板,也不要给《The origin and current situation of Weifang Kite Festival》的汇报套上灰暗阴沉的模板……

其次,请务必注意删除不属于我校的相关内容,例如,发现其他学校同学做的的PPT模板很好,套用了以后好不容易晚上糊弄完了内容却忘记了删除人家的校徽,或是在文中出现了其他非我校同学的姓名、班级等信息……

还有,模板有时还会给你带来\textbf{意料之外的变数},例如:\textbf{放映时的声音},尤其是很多教室的电脑默认音量为100\%的时候,一切切换声音、元素出现与消失声音都将震耳欲聋。因此,除非你已经提前实验过并且确信无疑声音毫无问题,不要在你的PPT中加入任何声音。

\subsection[校徽与学校标识的使用]{校徽与学校标识的使用}
学校校徽及图标等标识使用规范参见\uhref{https://www.sdsmu.edu.cn/4229/list.htm}{《山东第二医科大学VIS视觉识别系统手册》}(由校宣传部印发)

官方示例如下图所示,为4:3型,PPT以“潍医红”\footnotemark 为底色,中间内容部分以白色为基底,左上角有徽标+校名各一,其余照常。(个人建议建议仅加校徽而无底色)
\footnotetext{CYMK (10, 100, 100, 40) 或 RGB (154, 0, 4)。}

\begin{table}[H]
    \centering
    \includegraphics[height=200px]{ppt-内容.pdf}
\end{table}

\subsection[公式]{公式}
部分PPT需要向其中插入公式,在此敝人强烈建议额外加入一图片形式的公式(截图并插入)于原公式上一个图层,保证公式在低配置、低版本Office下仍可正常显示,避免因演示电脑的版本过低或配置不足导致原公式无法显示

\subsection[避免自动换页(自动计时)]{避免自动换页(自动计时)}
为使幻灯片在放映时能应对各突发情况,我极其建议取消排练的自动放映时间,代之以手动控制。该设置可在“幻灯片放映”→“使用计时”处,或“幻灯片放映”菜单→设置幻灯片放映→“推进幻灯片-手动”处调整

\subsection[善用批注与演讲者视图]{善用批注与演讲者视图}
PPT提供了在多屏幕下的批注与演讲者视图功能,在存在多个显示设备的情况下,可以指定其中一个设备显示演讲者的批注、放映时间与下一张幻灯片的预览视图(相当于后台模式),其余设备正常显示当前的幻灯片,便于演讲人梳理思路、衔接不同页面。(建议自行练习后使用)该设置可在“幻灯片放映”菜单中找到

\subsection[仅嵌入部分字体]{仅嵌入部分字体}
该设置旨在帮助用户在不同电脑上正确显示相同的字体,即便其未安装也一样。该设置可以通过“文件”菜单→更多→选项→保存→共享此文稿时保持保真度→“嵌入字体(仅嵌入演示文稿中使用的部分字符)”处进行设置。如遇嵌入失败,请检查该字体是否严重过大(≥20㎆)或为“.ttc”格式。如存在上述问题,请拷贝字体到U盘内并在演示电脑上安装。

\section[附件]{附件}
请自行提取,方法略。

\begin{table}[H]
    \centering
    \fbox{\includegraphics[width=.95\linewidth]{徽标-校名.pdf}}\\[10pt]
    \fbox{\includegraphics[angle=90,width=.95\linewidth]{校徽-校名-竖.pdf}}
\end{table}